\documentclass[aspectratio=1610]{beamer}
% \documentclass{article}
% \usepackage{beamerarticle}
\usetheme[navigation,footline=fullsinglelight,pagenumbers=full,transition=push,headline=light]{TUD}

% for handout
% \usepackage{pgfpages}
% \pgfpagesuselayout{4 on 1}[a4paper,landscape,border shrink=2mm]

\usepackage[T1]{fontenc}
\usepackage[utf8]{inputenc}
\usepackage[english]{babel}

\usepackage[all=normal,floats,wordspacing,leading,charwidths,tracking]{savetrees}
\usepackage{tudcolors}
\usepackage{mathtools}
\usepackage{amsmath}
\usepackage{amssymb}
\usepackage{xparse}
\usepackage{csquotes}
\usepackage{scalerel}
\usepackage{stmaryrd}

\usepackage{csvsimple}
\usepackage{filecontents}
\usepackage{pgfplots}
\usepackage{tikz}
\usetikzlibrary{positioning}
\usetikzlibrary{decorations.pathmorphing}
\usetikzlibrary{decorations.pathreplacing}
\usetikzlibrary{calc}
\usetikzlibrary{arrows}

\usepackage{multirow}
\usepackage{comment}
\usepackage{rotating}

\usepackage{xspace}
% \usepackage{bm}
% \def\bm{}

\let\Bbb\undefined

\def\Alphabet{A,B,C,D,E,F,G,H,I,J,K,L,M,N,O,P,Q,R,S,T,U,V,W,X,Y,Z}
\def\alphabet{a,b,c,d,e,f,g,h,i,j,k,l,m,n,o,p,q,r,s,t,u,v,w,x,y,z}
\newcommand{\mydef}[2]{\expandafter#1\expandafter{\csname#2\endcsname}}
\newcommand{\myforcsvlist}[2]{\expandafter\forcsvlist\expandafter#1\expandafter{#2}}

\newcommand{\defmc}[1]{\mydef{\newcommand}{#1mc}{\ensuremath{\mathcal{#1}}\xspace}}
\newcommand{\defmf}[1]{\mydef{\newcommand}{#1mf}{\ensuremath{\mathfrak{#1}}\xspace}}
\newcommand{\defbb}[1]{\mydef{\newcommand}{#1bb}{\ensuremath{\mathbb{#1}}\xspace}}
\newcommand{\defbf}[1]{\mydef{\newcommand}{#1bf}{\ensuremath{\mathbf{#1}}\xspace}}
\newcommand{\deful}[1]{\mydef{\newcommand}{#1ul}{\ensuremath{\underline{#1}}\xspace}}

\myforcsvlist{\defmc}{\Alphabet}
\myforcsvlist{\defmf}{\Alphabet}
\myforcsvlist{\defbb}{\Alphabet}
\myforcsvlist{\defbf}{\Alphabet}
\myforcsvlist{\defbf}{\alphabet}
\myforcsvlist{\deful}{\Alphabet}

\let\Xmc\undefined

\title
  [TUD Beamer Theme]
  {A theme for the LaTeX beamer class in corporate design of Technische Universität Dresden}
\subtitle{Some Exhilarating New Results Showing that the Earth is not a Ball}
\author{Francesco Kriegel}
\institute[TU Dresden]{Technische Universität Dresden}
\date[FCA4AI\,@\,ECAI 2016]{FCA4AI\,@\,ECAI, The Hague, 30 August 2016}

\begin{document}

\maketitle

\section{Section 1}

\begin{frame}{A}{Subtitle}
  \begin{enumerate}
    \item
      foo
    \item
      bar
    \item
      baz
  \end{enumerate}
  some comments
  \begin{itemize}
    \item
      an item
    \item
      another item
    \item
      yet another item
  \end{itemize}
\end{frame}

\begin{frame}
  A frame without a title.  Does it work?
\end{frame}

\begin{frame}{Just a title, but no subtitle}
  
  Some introductory text.
  
  \pause
  
  \begin{definition}
    This is an exemplary definition.
  \end{definition}
  
  \pause
  
  Some concluding remarks.
  
\end{frame}

\begin{frame}{C}{Subtitle}
  bla bla bla
  \begin{equation*}
    \int_{X}f\;d\mu\quad0123456789
  \end{equation*}
\end{frame}

\begin{frame}{D}{Subtitle}
  bla bla bla
  
%   \begin{posterbox}{An exemplary posterbox}
%     foo bar baz
%   \end{posterbox}
  
  \begin{definition}[Comparison with amsthm]
    foo bar baz
  \end{definition}
\end{frame}

\section{Section 2}

\begin{frame}{E}{Subtitle}
  bla bla bla
\end{frame}

\begin{frame}{F}{Subtitle}
  bla bla bla
\end{frame}

\begin{frame}{G}{Subtitle}
  bla bla bla
\end{frame}

\section{Section 3}

\begin{frame}{H}{Subtitle}
  bla bla bla
\end{frame}

\begin{frame}{I}{Subtitle}
  bla bla bla
\end{frame}
 
\end{document}